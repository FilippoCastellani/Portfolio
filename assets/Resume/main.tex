%%%%%%%%%%%%%%%%%%%%%%%%%%%%%%%%%%%%%%%%%
% Developer CV
% LaTeX Class
% Version 2.0 (12/10/23)
%
% This class originates from:
% http://www.LaTeXTemplates.com
%
% Authors:
% Omar Roldan
% Based on a template by  Jan Vorisek (jan@vorisek.me)
% Based on a template by Jan Küster (info@jankuester.com)
% Modified for LaTeX Templates by Vel (vel@LaTeXTemplates.com)
%
% License:
% The MIT License (see included LICENSE file)
%
%%%%%%%%%%%%%%%%%%%%%%%%%%%%%%%%%%%%%%%%%

%----------------------------------------------------------------------------------------
%	PACKAGES AND OTHER DOCUMENT CONFIGURATIONS
%----------------------------------------------------------------------------------------

\documentclass[9pt]{developercv} % Default font size, values from 8-12pt are recommended
\usepackage{multicol}
\setlength{\columnsep}{0mm}
%----------------------------------------------------------------------------------------
\usepackage{lipsum}  


\begin{document}

%----------------------------------------------------------------------------------------
%	TITLE AND CONTACT INFORMATION
%----------------------------------------------------------------------------------------

\begin{minipage}[t]{0.5\textwidth} 
	\vspace{-\baselineskip} % Required for vertically aligning minipages
	
	{ \fontsize{16}{20} \textcolor{black}{\textbf{\MakeUppercase{FILIPPO CASTELLANI}}}} % First name
	
	\vspace{6pt}
	
	{\Large PhD Candidate in Biorobotics\\ \\\footnotesize{Birth date: 5th Feb 1999}} % Career or current job title
\end{minipage}
\hfill
\begin{minipage}[t]{0.2\textwidth} % 20% of the page width for the first row of icons
	\vspace{-\baselineskip} % Required for vertically aligning minipages
	
	% The first parameter is the FontAwesome icon name, the second is the box size and the third is the text
	\icon{Globe}{11}{\href{https://filippocastellani.github.io/Portfolio/}{Portfolio Website}}\\ 
    \icon{Phone}{11}{+39 3482740303}\\
    \icon{MapMarker}{11}{Paris, France}\\
	
\end{minipage}
\begin{minipage}[t]{0.27\textwidth} % 27% of the page width for the second row of icons
	\vspace{-\baselineskip} % Required for vertically aligning minipages
	
	\icon{Envelope}{11}{\href{mailto:castellanifilippo5@gmail.com}{castellanifilippo5@gmail.com}}\\	
    \icon{Github}{11}{\href{https://github.com/FilippoCastellani}{github.com/FilippoCastellani}}\\
    \icon{LinkedinSquare}{11}{\href{https://www.linkedin.com/in/filippo-castellani}{/in/filippo-castellani}}\\    
    
\end{minipage}


%----------------------------------------------------------------------------------------
%	INTRODUCTION, SKILLS AND TECHNOLOGIES
%----------------------------------------------------------------------------------------

\begin{minipage}[t]{0.55\textwidth}
    \cvsect{Summary}
	\vspace{-6pt}
 
    %Dummy text
    Biomedical Engineering student driven by a profound interest in Neuroscience, Brain-Computer Interfaces (BCI), and AI-driven solutions for rehabilitation.\\  
    I have a solid engineering background, reinforced by hands-on experience acquired in clinical settings and conducting experiments with technical instruments and biological tissues. With my work, I aim to make significant contributions to the field of biomedical solutions, enhancing the well-being of patients affected by neural disabilities.
	
\end{minipage}
\hfill % Whitespace between
\begin{minipage}[t]{0.405\textwidth}
    \cvsect{Technical Expertise}
    \vspace{-6pt}
    
    \begin{minipage}[t]{0.27\textwidth}
        \textbf{Coding \\ Languages:}
    \end{minipage}
    \hfill
    \begin{minipage}[t]{0.70\textwidth}
      Python, Matlab, C++, \\ JavaScript, SQL, HTML, CSS.
    \end{minipage}
    \vspace{4mm}
    
    \begin{minipage}[t]{0.27\textwidth}
        \textbf{Electro-\\ Physiology:}
    \end{minipage}
    \hfill
    \begin{minipage}[t]{0.70\textwidth}
      EEG acquisition,\\ EMG acquisition,\\ FES stimulation,\\ Neural population recording using Multi-Electrode Arrays.
    \end{minipage}
    
\end{minipage}


%----------------------------------------------------------------------------------------
%	EXPERIENCE
%----------------------------------------------------------------------------------------
\vspace{-1 pt}
\cvsect{Experience}
\begin{entrylist}
	\entry
        {10/2023 -- 04/2024}
		{ Research internship (Master Thesis)}
		{\textcolor{blue}{{\href{https://www.institut-vision.org/en/}{Institut de La Vision (Sorbonne Université), Paris, FR}}}}
		{\vspace{-10pt}
        \begin{itemize}[noitemsep,topsep=0pt,parsep=0pt,partopsep=0pt, leftmargin=-1pt]
            \item Research visual information encoding in biological neural networks, in particular, in the retina.
            \item Study Retinal Ganglion Cells response through Multi-Electrode Array recording on ex-vivo mouse retina.
            \item Devise and carry out experiments to probe and explore visual information encoding mechanisms during biomimetic stimulation.
            \item Analyse the electrophysiological response of neurons to uncover the mechanisms behind poorly understood color encoding.
            
        \end{itemize} 
        
        }
        
	\entry
		{11/2021 -- 09/2023}
		{Neurotechnology Researcher \\\footnotesize{RECOMMENCER Project: \textit{Currently undergoing \textcolor{blue}{\href{https://clinicaltrials.gov/study/NCT05511207}{clinical trial - NCT05511207}}}}}
		{\textcolor{blue}{\href{https://www.hsantalucia.it/laboratorio-immagini-neuroelettriche-interfacce-cervello-computer}{S.Lucia Foundation IRCCS, Rome, IT}}}
		{\vspace{-10pt}
        \begin{itemize}[noitemsep,topsep=0pt,parsep=0pt,partopsep=0pt, leftmargin=-1pt]
            \item Implementation of a bidirectional Brain Computer Interface (BCI) for upper limb rehabilitation in post-stroke subjects.
            \item Develop real-time algorithm to perform electroencephalographic (EEG), electromyographic (EMG) signals analysis and eventually extracting Corticomuscular Coherence features.
            \item Code and test the features classification logic that relays neurofeedback through Functional Electrical Stimulation (FES).
            \item Design and integrate information processing modules in a cohesive data pipeline, from acquisition to sensorial neurofeedback.
            \item Create the therapist interface for rehabilitation session management. Write documentation and software version management.
        \end{itemize} 
        }
\end{entrylist}


%----------------------------------------------------------------------------------------
%	Publications
%----------------------------------------------------------------------------------------
\vspace{-15 pt}
\cvsect{Publications}
\begin{entrylist}
    \entry
		{2022}
		{Cortico-Muscular Coupling to Control a Hybrid Brain-Computer Interface for Upper Limb Motor Rehabilitation: A Pseudo-Online Study on Stroke Patients.}
		{\textcolor{blue}{\href{https://doi.org/10.3389/fnhum.2022.1016862}
{Front. Human Neuroscience 2022, 16, 1016862.}}}
		{de Seta, V.; Toppi, J.; Colamarino, E.; Molle, R.; Castellani, F.; Cincotti, F.; Mattia, D.; Pichiorri, F.}
    
\end{entrylist}


%----------------------------------------------------------------------------------------
%	Projects
%----------------------------------------------------------------------------------------
\vspace{-15 pt}
\cvsect{Projects}
\begin{entrylist}
    \entry
		{Up-to-Date}
		{My projects}
		{\textcolor{blue}{\href{https://filippocastellani.github.io/Portfolio/}{Portfolio}}}
		{ \icon{Rocket}{11}{Collection of recent, past and ongoing projects.} }
\end{entrylist}

%----------------------------------------------------------------------------------------
%	EDUCATION
%----------------------------------------------------------------------------------------
\vspace{-15 pt}
\cvsect{Education}
\begin{entrylist}
    \entry
		{9/2021 - today \\\footnotesize{scholarship holder}}
		{MSc Biomedical Engineering - Technologies for Electronics }
		{Politecnico di Milano, Milan, IT}
		{Thesis: currently under development}
    \entry
		{9/2018 - 10/2021 \\\footnotesize{scholarship holder}}
		{BSc Clinical Engineering}
		{Sapienza Università di Roma, Rome, IT}
		{Thesis: Coherence-Based BCI for Rehabilitation: Feature Extraction and Experimental Assessment
   \begin{itemize}[noitemsep,topsep=0pt,parsep=0pt,partopsep=0pt, leftmargin= 10pt]
            \item Perform research on state of-the-art use of coherence-based BCI. 
            \item Implementing via Python, a feature extraction algorithm executable within the OpenVibe Software framework.
            \item Conduct laboratory test of features extraction from non-pathological subjects.

        \end{itemize} 
             }
	\entry
		{9/2011 - 9/2019}
		{Jazz Drum [2011-2015] and Electronic Music [2018-2019]}
		{Conservatory of Music Santa Cecilia, Rome, IT}
		{ \begin{itemize}[noitemsep,topsep=0pt,parsep=0pt,partopsep=0pt, leftmargin= 10pt]
            \item Completed coursework in composition, music theory, application of signal theory to sound design. 
            \item Proficient in solfége, pianoforte, Jazz drumming techniques, as well as orchestral performance.
            \end{itemize}
        }
\end{entrylist}

%----------------------------------------------------------------------------------------
%	LANGUAGES
%----------------------------------------------------------------------------------------
\vspace{-15 pt}
	\cvsect{Languages}
    \vspace{-6pt}
    
    \hspace{26mm} \textbf{English}: B2 (Cambridge), \textbf{French}: C1 (Alliance Française), \textbf{Spanish}: B1 (Istituto Cervantes), \textbf{ Italian}: native.

%----------------------------------------------------------------------------------------

\end{document}
